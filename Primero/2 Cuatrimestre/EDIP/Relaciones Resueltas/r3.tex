\documentclass{article}
\usepackage[utf8]{inputenc}

\title{Relación 3}
\author{Daniel Alconchel Vázquez }
\date{8 de Abril de 2020}

\begin{document}

\maketitle
\newpage
\begin{raggedright}
\begin{bf}
1. Durante un año, las personas de una ciudad utilizan 3 tipos de transporte: metro(M), autobús(A), y coche particulas(C). Las probabilidades de que durante el año hayan usado uno u otros transportes son:

\smallskip
M: 0.3; A: 0.2; C: 0.15; M y A: 0.1; M y C:0.05; A y C: 0.06; M,A y C: 0.01

\smallskip
Calcular las siguientes probabilidades:

\smallskip
a) que una persona viaje en metro y no en autobús;

\smallskip
b) que una persona tome al menos dos medios de transporte;

\smallskip
c) que una persona viaje en metro o en coche, pero no en autobús;

\smallskip
d) que viaje en metro, o bien en autobús y en coche;

\smallskip
e) que una persona vaya a pie
\end{bf}

\begin{rm}
\bigskip
a) \begin{equation}
    P(M\cap \overline{A}) = P(M-A) = P(M) - P(M\cap A) = 0.2 
\end{equation}

\smallskip
b) \begin{equation}
    P(M\cap A) + P(M\cap C) - P(A\cap M\cap C) + P(A\cap C) - P(A\cap M\cap C) = 0.19 
\end{equation}

\smallskip
Hay que restarle $P(A\cap M\cap C)$ porque dicha probabilidad esta contenida en la suma de las anteriores.

\bigskip
c)

\smallskip
\begin{equation}
     P(M\cap C)\cap \overline{A}) = P((M\cap \overline{A}) \cap P(C\cap \overline{A})) = \\
    = P(M\cap \overline{A}) + P(C\cap \overline{A}) - P(\overline{A}\cap M\cap C) = \\
    = P(M-A) + P(C-A) - P((M\cap C)-A) = \\
    = P(M) - P(M\cap A) + P(C) - P(C\cap A) - P(M\cap C) + P(M\cap C\cap A) = 0.25
\end{equation}

\smallskip
d) \begin{equation}
    P(M\cap(A\cap C)) = P(M) + P(A\cap C) - P(M\cap A\cap C) = 0.35
\end{equation}

\smallskip
e) No se puede calcular, ya que dicho suceso no pertenece al espacio muestral.
\end{rm}
\end{raggedright}

\begin{bf}
\newpage
Sean A,B,C tres sucesos de un espacio probabilístico (\Omega, $\textit{A}$, P). \\
$P(A) = 0,4, P(B) = 0,2, P(C) = 0,3, P(A\cap B) = 0,1, P(A\cup B)\cap C = \emptyset$. Calcular las probabilidades de los siguientes sucesos:

\smallskip
a) sólo ocurre A.

\smallskip
b) ocurren los tres sucesos.

\smallskip
c) ocurren A y B pero no C.

\smallskip
d) por los menos dos ocurren.

\smallskip
e) ocurren dos y no más.

\smallskip
f) no ocurren dos y no más.

\smallskip
g) ocurre por lo menos uno.

\smallskip
h) ocurre sólo uno.

\smallskip
i) no ocurre ninguno.
\end{bf}

\begin{rm}
\bigskip
a)\begin{equation}
    P(A\cap \overline{B}) = P(A) - P(A\cap B) = 0,4 - 0,1 = 0,3
\end{equation}

\smallskip
b)\begin{equation}
    P((A\cup B)\cap C) = \emptyset \implie P(A\cap C \cup B\cap C) = \emptyset \implies P(A\cap C) \emptyset, y,P(B\cap C)\emptyset \implies P(A\cap B\cap C) = \emptyset
\end{equation}

\smallskip
c)\begin{equation}
    P((A\cap B)\cup \overline{C}) = P(A\cap B) - P(A\cap B\cap C) = 0,1 - \emptyset = 0,1
\end{equation}

\smallskip
d)
\smallskip

\begin{multline*}
     P((A\cap B)\cup (A\cap C)\cup (B\cap C) = \\
     P((A\cap B)\cup (A\cap C)) + \cancel{P(B\cap C)} - P(\cancel{(A\cap B)}\cup (A\cap C)\cup (B\cap C)) = \\
     p(A\cap B) + \cancel{P(A\cap C)} - \cancel{P((A\cap B\cap C)} = 0,1
\end{multiline*}

\smallskip
e) Como la intersección de C con el resto de sucesos es vacío, la única opción es: $P(A\cap B) = 0,1$

\end{rm}

\end{document}